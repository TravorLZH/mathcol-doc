\section{Integral of the \textit{sinc} function}

Not every integral can be evaluated by \textbf{Newton-Leibniz formula}. Some of
them require certain tricks, and this integral is a perfect example:

\begin{equation}
	\int_0^\infty{\sin(x)\over x}dx
	\label{int-sinc}
\end{equation}

The function inside the integral of (\ref{int-sinc}) is called the
\textit{sinc} function, which is extremely useful in the field of signal and
image processing. To integrate this function, we could consider finding the
Laplace transform of this function:

\begin{equation}
	\int_0^\infty{\sin(x)\over x}dx=\mathcal{L}
	\left\{\sin(x)\over x\right\}(0)
	\label{sinc-laplace}
\end{equation}

Therefore, the integration of \textit{sinc} is divided into two steps:

\begin{itemize}
	\item Find the Laplace transform of \textit{sinc}
	\item Evaluate the Laplace transform at $s=0$
\end{itemize}

\subsection{Laplace transform of \textit{sinc}}

We denote the Laplace transform of \textit{sinc} as $F(s)$, conveniencing the
evaluation.

$$F(s)=\int_0^\infty{\sin(x)\over x}e^{-sx}dx$$
$$
\begin{aligned}
	F'(s)&={d\over ds}\int_0^\infty{\sin(x)\over x}e^{-sx}dx \\
	&=\int_0^\infty{\partial\over\partial s}{\sin(x)\over x}e^{-sx}dx \\
	&=-\int_0^\infty\sin(x)e^{-sx}dx=-\mathcal{L}\{\sin(x)\}
\end{aligned}
$$

Now we are to find the Laplace transform of sine function. It will take two
iterations of integration by parts if we evaluate the integral directly.
However, if we complexify it, turning $\sin(x)$ to $\Im\left(e^{ix}\right)$ in
other words, we can make use of the properties of exponential function so that
the process becomes easier.

$$
\begin{aligned}
	\mathcal{L}\{\sin(x)\}&=\int_0^\infty\Im\left(e^{ix}\right)e^{-sx}dx \\
	&=\Im\left(\int_0^\infty e^{ix-sx}dx\right)
	=\Im\left(\int_0^\infty e^{(i-s)x}dx\right) \\
	&=\Im\left(\left.e^{(i-s)x}\over i-s\right|_0^\infty\right) \\
	&=\Im\left(\lim_{x\to\infty}{e^{(i-s)x}\over i-s}-{1\over i-s}\right) \\
	&=\Im\left(1\over s-i\right)=\Im\left(s+i\over(s+i)(s-i)\right) \\
	&=\Im\left(s+i\over s^2+1\right)={1\over s^2+1}
\end{aligned}
$$

After plugging the result of this function, we simplify the expression of
$F'(s)$:

\begin{equation}
	F'(s)=-{1\over s^2+1}
	\label{sinc-deriv}
\end{equation}

Before trying to find the exact $F(s)$, we also need to consider its property,
especially its horizontal assymptote.

$$
\lim_{s\to\infty}F(s)=\int_0^\infty{\sin(x)\over x}
\lim_{s\to\infty}\left(e^{-sx}dx\right)=0
$$

Therefore, $F(s)$ can be represented by what was concluded in equation
(\ref{sinc-deriv}) via expressing it as a subtraction:

$$
\begin{aligned}
	F(s)&=F(s)-\lim_{b\to\infty}F(b) \\
	&=\int_\infty^sF'(t)dt=-\int_s^\infty F'(t)dt \\
	&=-\int_s^\infty{-1\over t^2+1}dt=\left.\arctan(t)\right|_s^\infty \\
	&={\pi\over 2}-s
\end{aligned}
$$

\subsection{Evaluate the Laplace transform of \textit{sinc} at $s=0$}

Since we have all the necessary equations calculated, we can now take down the
integral by simply plugging in that sacred zero.

$$
\begin{aligned}
	\int_0^\infty{\sin(x)\over x}dx&=\mathcal{L}\left\{\sin(x)\over x\right\}(0) \\
	&=F(0)={\pi\over2}-0={\pi\over2}
\end{aligned}
$$

As it turns out, the integral of \textit{sinc} from zero to infinity is half of
$\pi$.
