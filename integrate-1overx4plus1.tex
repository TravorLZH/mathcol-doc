\section{Integral of $\frac{1}{x^4+1}$}
When we first see function $\frac{1}{x^4+1}$, it always reminds us of the
derivative of arc tangent (which is $\frac{1}{x^2+1}$). Therefore we can solve
this using some intuitions of arc tangent. First step, we try to make it into
multiplication of polynomials in order to facilitate the integration:

$$
\begin{aligned}
	x^4+1
	&=(x^4+2x^2+1)-2x^2 \\
	&=(x^2+1)^2-(\sqrt2x)^2 \\
	&=(x^2-\sqrt2x+1)(x^2+\sqrt{2x}+1) \\
\end{aligned}
$$

It seems that we are now able to deal with it using partial fractions:

$$\frac{1}{x^4+1}=\frac{Ax+B}{x^2-\sqrt2x+1}+\frac{Cx+D}{x^2+\sqrt2x+1}$$

Since we know the numerator of the fraction in the left part is one, we can
make the addition of numerator equal to $1$:

$$
\begin{aligned}
	1=&(Ax+B)(x^2+\sqrt2x+1)+(Cx+D)(x^2-\sqrt2x+1) \\
	1=&(Ax^3+A\sqrt2x^2+Ax)+(Bx^2+B\sqrt2x+B)+ \\
	&(Cx^3-C\sqrt2x^2+Cx)+(Dx^2-D\sqrt2x+D) \\
	1=&(A+C)x^3+[\sqrt2(A-C)+B+D]x^2+[A+C+\sqrt2(B-D)]x+B+D
\end{aligned}
$$

We can turn this nasty polynomial equation into linear equations:

$$
\begin{aligned}
	\begin{cases}
		A+C=0 \\
		\sqrt2(A-C)+B+D=0 \\
		A+C+\sqrt2(B-D)=0 \\
		B+D=1 \label{e_d}
	\end{cases} \\
	\because B+D=1 \\
	\therefore \sqrt2(A-C)+1=0 \\
	\therefore A-C=-\frac{1}{\sqrt2} \\
	\because A+C=0 \\
	\therefore 2A=-\frac{1}{\sqrt2} \\
	\therefore A=-\frac{1}{2\sqrt2} \\
	\therefore C=\frac{1}{2\sqrt2} \\
	\because B+D=1,B-D=0 \\
	\therefore B=D=\frac{1}2 \\
	\begin{cases}
		A=-\frac{1}{\sqrt2} \\
		B=\frac{1}2 \\
		C=\frac{1}{\sqrt2} \\
		D=\frac{1}2 \\
	\end{cases} \\
	\therefore \frac{1}{x^4+1}=\frac{-\frac{1}{2\sqrt2}x+
	\frac{1}2}{x^2-\sqrt2x+1}+\frac{\frac{1}{2\sqrt2}x+
	\frac{1}2}{x^2+\sqrt2x+1}
\end{aligned}
$$

Since we split the one fraction into two fractions in which each denominator is
made of a second-power trinomial, so we can complete squares for integrating
for arc tangent.

\begin{eqnarray}
	\begin{aligned}
		x^2-\sqrt2x+1
		&=(x^2-\sqrt2x+\frac{1}2)+\frac{1}2 \\
		&=(x-\frac{1}{\sqrt2})^2+\frac{1}2
	\end{aligned} \label{eqn1}\\
	x^2+\sqrt2x+1=(x+\frac{1}{\sqrt2})^2+\frac{1}2 \label{eqn2}
\end{eqnarray}

Now we can integrate since everything is ready. The integration starts with
partial fractions and then plugging in \ref{eqn1} and \ref{eqn2}

$$
\begin{aligned}
	\int\frac{dx}{x^4+1}&=\int\frac{-\frac{1}{2\sqrt2}x+
	\frac{1}2}{x^2-\sqrt2x+1}dx+\int\frac{\frac{1}{2\sqrt2}x+
	\frac{1}2}{x^2+\sqrt2x+1}dx \\
	&=\frac{1}{4\sqrt2}\left(\int\frac{-2x+2\sqrt2}{(x-\frac{1}{\sqrt2})^2
	+\frac{1}2}dx+\int\frac{2x+2\sqrt2}{(x+\frac{1}{\sqrt2})^2+\frac{1}2}dx
	\right)
\end{aligned}
$$

Now we start dealing with them with the first part:

$$
\begin{aligned}
	\int\frac{-2x+2\sqrt2}{(x-\frac{1}{\sqrt2})^2+\frac{1}2}dx
	&=-\int\frac{2x}{(x-\sqrt2)^2+\frac{1}2}dx+
	\int\frac{2\sqrt2}{(x-\sqrt2)^2+\frac{1}2}dx
\end{aligned}
$$

It seems like a \textbf{$u$-substitution} is applicable in the first integral,
so we let $u=(x-\sqrt2)^2+\frac{1}2$, so $du=2xdx$. Then the whole thing is
transformed into this:

$$
\begin{aligned}
	-\int\frac{2x}{(x-\sqrt2)^2+\frac{1}2}dx+
	\int\frac{2\sqrt2}{(x-\sqrt2)^2+\frac{1}2}dx
	&=-\int\frac{du}u+
	\int\frac{2\sqrt2}{(x-\sqrt2)^2+\frac{1}2}dx \\
	&=-\ln(u)+
	\int\frac{2\sqrt2}{(x-\sqrt2)^2+\frac{1}2}dx \\
	&=-\ln(x^2-\sqrt2x+1)+
	\int\frac{2\sqrt2}{(x-\sqrt2)^2+\frac{1}2}dx \\
	&=-\ln(x^2-\sqrt2x+1)+
	\int\frac{2\sqrt2}{(x-\frac{1}{\sqrt2})^2
	+(\frac{1}{\sqrt2})^2}dx \\
	&=-\ln(x^2-\sqrt2x+1)+\sqrt2\arctan(\sqrt2x-1)
\end{aligned}
$$

The integration of the second integral is very similar, at last the indefinite
integral of $\frac1{x^4+1}$ is:

$$
\frac1{4\sqrt2}\left(\ln({x^2+\sqrt2x+1\over x^2-\sqrt2x+1})+
2\arctan(\sqrt2x+1)+2\arctan(\sqrt2x-1)\right)+C
$$
