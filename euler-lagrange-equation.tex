\section{Intuition of Euler-Lagrange equation}

Euler-Lagrange equation is a \textbf{partial differential equation} fundamental
to the study of the \textbf{calculus of variations}, especially in the field of
optimizing a \textbf{functional}. There are many methods to derive this
equation rigourously, but this section will give readers a basic understanding
of where this equation comes from.

\subsection{Functional}

According to calculus of variations, a typical functional is defined as:

\begin{equation}
	y\mapsto J[y]=\int_{x_0}^{x_1}F(x,y,\dot{y})dx
	\label{functional-def}
\end{equation}

where $y$ is a function dependent on $x$. Arc length, for instance, is a
typical functional:

$$
y\mapsto L[y]=\int_{x_0}^{x_1}\sqrt{1+\dot{y}^2}dx
$$

in which $F$ is called the \textbf{core} of the functional.

\subsection{Optimize functional}

In single-variable calculus, we know that a function reaches an extremum if its
derivative becomes zero, and this is same for functionals. Since the solution
of the optimization of a problem is a function instead of a real number, we
assume $y$ is the optimal function and define $f$ as follows:

\begin{equation}
	f=y+\varepsilon\mu
	\label{var-y}
\end{equation}

In equation (\ref{var-y}), $\mu$ is an arbitrary function satisfying
$\mu(x_0)=\mu(x_1)=0$, and $\varepsilon\in\mathbb{R}$. Since $y$ denotes the
result of optimization of $J[f]$, the function $N(\varepsilon)=J[f]$ reaches
its extremum at $\varepsilon=0$. Therefore, we can deduce:

\begin{equation}
	\left.dN\over d\varepsilon\right|_{\varepsilon=0}=0
	\label{optimal-functional}
\end{equation}

In order to find out $y$, we first need to express the derivative in
(\ref{optimal-functional}). According to \textbf{Leibniz's rule}, we can swap
the differential operator and integral if the limits of the integral are
not depending on $\varepsilon$.

$$
\begin{aligned}
	{dN\over d\varepsilon}
	&={d\over d\varepsilon}\int_{x_0}^{x_1}F(x,f,\dot{f})dx \\
	&=\int_{x_0}^{x_1}{d\over d\varepsilon}F(x,f,\dot{f})dx
\end{aligned}
$$

Since $F$ is a multivariable function, we take its total derivative with
respect to $\varepsilon$. Since $x$ does not depend on $\varepsilon$,
$dx/d\varepsilon$ is automatically set to zero.

$$
\begin{aligned}
	{dN\over d\varepsilon}
	&=\int_{x_0}^{x_1}{\partial F\over\partial x}
	{dx\over d\varepsilon}+{\partial F\over\partial f}
	{df\over d\varepsilon}+{\partial F\over\partial\dot{f}}
	{d\dot{f}\over d\varepsilon}dx \\
	&=\int_{x_0}^{x_1}{\partial F\over\partial f}{df\over d\varepsilon}
	+{\partial F\over\partial\dot{f}}{d\dot{f}\over d\varepsilon}dx
\end{aligned}
$$

According to the definition of $f$ in (\ref{var-y}), $df/d\varepsilon$ is
simply $\mu$, and $d\dot{f}/d\varepsilon=\dot{\mu}$.

$$
{dN\over d\varepsilon}=\int_{x_0}^{x_1}\mu{\partial F\over\partial f}
+\dot{mu}{\partial F\over\partial\dot{f}}dx
$$

Using the technique of integration by parts, $\mu$ became factored out. Because
$\mu(x_0)=\mu(x_1)=0$, there is nothing left outside the integral.

$$
\begin{aligned}
	{dN\over d\varepsilon}
	&=\int_{x_0}^{x_1}\mu{\partial F\over\partial f}dx+
	\left.\mu{\partial F\over\partial\dot{f}}\right|_{x_0}^{x_1}
	-\int_{x_0}^{x_1}\mu{d\over dx}{\partial F\over\partial\dot{f}}dx \\
	&=\int_{x_0}^{x_1}\mu\left({\partial F\over\partial f}-
	{d\over dx}{\partial F\over\partial\dot{f}}\right)dx
\end{aligned}
$$

According to (\ref{optimal-functional}) and \ref{var-y}, we replace
$dN/d\varepsilon$, $f$, and $\dot{f}$ with zero, $y$, and $\dot{y}$
respectively.

$$
0=\int_{x_0}^{x_1}\mu\left({\partial F\over\partial y}-{d\over dx}
{\partial F\over\partial\dot{y}}\right)dx
$$

According to the \textbf{fundamental lemma of calculus of variations}, we know
that the coefficient of $\mu$ must be zero if the integral is zero.

$$
{\partial F\over\partial y}-{d\over dx}{\partial F\over\partial\dot{y}}=0
$$

Therefore, we obtain the Euler-Lagrange equation:

\begin{equation}
	{\partial F\over\partial y}={d\over dx}{\partial F\over\partial\dot{y}}
	\label{el-def}
\end{equation}

\subsection{Shortcut for specific functional}

If the core $F$ of a functional does not depend on $x$ ($F=F(y,\dot{y})$), a
simpler approach is possible without taking the derivative of $\partial
F/\partial\dot{y}$, facilitating the whole optimization problem.

\begin{flushleft}
	Step 1: Eliminate the differential operator via integrating factor
\end{flushleft}

If we multiply (\ref{el-def}) by $\dot{y}$, we can formulate:

$$
{\partial F\over\partial y}\dot{y}=
{d\over dx}{\partial F\over\partial\dot{y}}\dot{y}
$$

According to product rule of differential calculus, the right part of the
equation is equivalent to:

$$
{d\over dx}\left({\partial F\over\partial\dot{y}}\dot{y}\right)
-\ddot{y}{\partial F\over\partial\dot{y}}
$$

As a result, we can gather a dot product in the left side.

$$
\left({\partial F\over\partial y}\hat{\i}
+{\partial F\over\partial\dot{y}}\hat{\j}\right)^\top
\left({dy\over dx}\hat{\i}+{d\dot{y}\over dx}\hat{\j}\right)
={d\over dx}\left({\partial F\over\partial\dot{y}}\dot{y}\right)
$$

We can discover that the left side is the total derivative of $F$ with respect
to $x$, so we find the antiderivative of the both side in order to eliminate
differential operator.

$$
F(y,\dot{y})+C={\partial F\over\partial\dot{y}}\dot{y}
$$

Therefore, when optimizing functionals with cores only depending on $y$ and
$\dot{y}$, we can use the following equation instead of the Euler-Lagrange
equation:

$$
F(y,\dot{y})-\dot{y}{\partial F\over\partial\dot{y}}=C
$$

where $C$ is a constant.
