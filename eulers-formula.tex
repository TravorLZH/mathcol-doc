\section{Derivation of Euler's formula}

Euler's formula is a basic tool that is widely used in the field of complex
analysis. It can directly tell that multiplication of complex numbers indicates
rotations.

In this section, Euler's formula will be proved using the \textbf{Maclaurin
expansion} of the following three function.

\begin{equation}
	e^x=\sum_{n=0}^\infty{x^n\over n!}
	=1+x+{x^2\over2!}+{x^3\over3!}+\dots
	\label{taylor-exp}
\end{equation}
\begin{equation}
	\sin(x)=\sum_{n=0}^\infty{(-1)^n\over(2n+1)!}x^{2n+1}
	=x-{x^3\over3!}+{x^5\over5!}-{x^7\over7!}+\dots
	\label{taylor-sin}
\end{equation}
\begin{equation}
	\cos(x)=\sum_{n=0}^\infty{(-1)^n\over(2n)!}x^{2n}
	=1-{x^2\over2!}+{x^4\over4!}-{x^6\over6!}+\dots
	\label{taylor-cos}
\end{equation}

First, if we substitute $x$ with $i\theta$ in the exponential function.
Therefore, the corresponding Maclaurin series can be written like this:

$$
\begin{aligned}
	e^{i\theta}
	&=1+i\theta+{i^2\theta^2\over2!}+{i^3\theta^3\over3!}+\dots \\
	&=1+i\theta-{\theta^2\over2!}-{i\theta^3\over3!}+{\theta^4\over4!}
	+\dots \\
	&=1-{\theta^2\over2!}+{\theta^4\over4!}-{\theta^6\over6!}+\dots
	+i\left(\theta-{\theta^3\over3!}+{\theta^5\over5!}-{\theta^7\over7!}
	+\dots\right)
\end{aligned}
$$

By mathematical induction, one can easily see that the real component of this
expression matches the pattern of equation (\ref{taylor-cos}) and the imaginary
part matches (\ref{taylor-sin}). Eventually, we inducted Euler's formula.

$$e^{i\theta}=\cos\theta+i\sin\theta$$

When substituting $\theta$ with $\pi$, the formula becomes \textbf{Euler's
identity}, which puts two transcendental numbers, two units of complex numbers,
and zero in a single equation.

$$e^{i\pi}+1=0$$
